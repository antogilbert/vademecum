\section{I declinazione}

\begin{table}[h!]
\centering
    \begin{tabular}{|c|l|l|}
        \hline
        \textbf{Caso} & \textbf{Singolare} & \textbf{Plurale} \\
        \hline
        Nom. & ros -\textbf{a}  & ros -\textbf{ae} \\
        \hline
        Gen. & ros -\textbf{ae} & ros -\textbf{arum} \\
        \hline
        Dat. & ros -\textbf{ae} & ros -\textbf{is} \\
        \hline
        Acc. & ros -\textbf{am} & ros -\textbf{as} \\
        \hline
        Voc. & ros -\textbf{a}  & ros -\textbf{ae} \\
        \hline
        Abl. & ros -\textbf{a}  & ros -\textbf{is} \\
        \hline
    \end{tabular}
\end{table}

\subsection{Eccezioni}
\subsubsection*{Genitivo singolare in -as}
\textit{Familia} con \textit{mater}, \textit{pater}, \textit{filius}, \textit{filia}: \textit{Pater familias} - Padre di famiglia

\subsubsection*{Genitivo plurale in -um}
\begin{itemize}
    \item \textit{Amphora}, \textit{drachma} : navis duum milium amphorum - nave di 2000 anfore
    \item Composti in -\textit{cola} e -\textit{gena}: caelicola, caelicolum / terrigena, terrigenum
    \item Patronimici
    \item Nomi di popoli: Aeneadae, Aeneadum
\end{itemize}

\subsubsection*{Dativo e ablativo pluarle in -abus}
Quando accompagnati dai corrispondenti nomi maschili per evitare confusione di generi. E.g. \textit{deis et deabus}, \textit{filis et filiabus}

\subsubsection*{Pluralia tantum}
Alcuni nomi sono privi del singolare. Quando tali nomi sono soggetto della frase, anche il verbo va al plurale.
Tra i pi\`u importanti:
\textit{deliciae, divitiae, epulae, exsequiae, feriae, indutiae, insidiae, minae, Kalendae, Nonae,
nuptiae, Athenae, Syracusae}.

\subsubsection*{Nomi con significato diverso tra singolare e plurale}
Tra i pi\`u importanti:
\begin{table}[h!]
    \begin{minipage}{.4\linewidth}
        \centering
        \begin{tabular}{|l|l|}
            \hline
            \textbf{Latino} & \textbf{Italiano} \\
            \hline
            copia & abbondanza \\
            \hline
            littera & lettera dell'alfabeto\\
            \hline
            opera & lavoro, attivit\`a \\
            \hline
            vigilia & veglia \\
            \hline
        \end{tabular}
        \caption{Singolare}
    \end{minipage}
    \hfill
    \begin{minipage}{.4\linewidth}
        \centering
        \begin{tabular}{|l|l|}
            \hline
            \textbf{Latino} & \textbf{Italiano} \\
            \hline
            copiae & milizie \\
            \hline
            litterae & letteratura, lettera\\
            \hline
            operae & operai \\
            \hline
            vigiliae & sentinelle \\
            \hline
        \end{tabular}
        \caption{Plurale}
    \end{minipage}
\end{table}
\clearpage
