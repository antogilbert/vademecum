\section{II declinazione}
\subsection{Nomi in -us e -um}
\begin{table}[h!]
    \begin{minipage}{.4\linewidth}
        \centering
        \begin{tabular}{|c|l|l|}
            \hline
            \textbf{Caso} & \textbf{Singolare} & \textbf{Plurale} \\
            \hline
            Nom. & lup -\textbf{us} & lup -\textbf{i} \\
            \hline
            Gen. & lup -\textbf{i}  & lup -\textbf{orum} \\
            \hline
            Dat. & lup -\textbf{o}  & lup -\textbf{is} \\
            \hline
            Acc. & lup -\textbf{um} & lup -\textbf{os} \\
            \hline
            Voc. & lup -\textbf{e}  & lup -\textbf{i} \\
            \hline
            Abl. & lup -\textbf{o}  & lup -\textbf{is} \\
            \hline
        \end{tabular}
        \caption{M \& F}
    \end{minipage}
    \hfill
    \begin{minipage}{.4\linewidth}
        \centering
        \begin{tabular}{|c|l|l|}
            \hline
            \textbf{Caso} & \textbf{Singolare} & \textbf{Plurale} \\
            \hline
            Nom. & bell -\textbf{um} & bell -\textbf{a} \\
            \hline
            Gen. & bell -\textbf{i}  & bell -\textbf{orum} \\
            \hline
            Dat. & bell -\textbf{o}  & bell -\textbf{is} \\
            \hline
            Acc. & bell -\textbf{um} & bell -\textbf{a} \\
            \hline
            Voc. & bell -\textbf{um} & bell -\textbf{a} \\
            \hline
            Abl. & bell -\textbf{o}  & bell -\textbf{is} \\
            \hline
        \end{tabular}
        \caption{N}
    \end{minipage}
\end{table}

\subsection{Nomi in -er e -ir}
I nomi terminanti in -\textit{ir} conservano sempre la i nella terminazione (tab. \ref{tab:conservazione_ii_decl}).
I nomi terminanti in -\textit{er} possono conservarla (tab. \ref{tab:conservazione_ii_decl}) o perderla
(tab. \ref{tab:perdita_ii_decl}).
\begin{table}[h!]
    \begin{minipage}{.4\linewidth}
        \centering
        \begin{tabular}{|c|l|l|}
            \hline
            \textbf{Caso} & \textbf{Singolare} & \textbf{Plurale} \\
            \hline
            Nom. & puer              & puer -\textbf{i} \\
            \hline
            Gen. & puer -\textbf{i}  & puer -\textbf{orum} \\
            \hline
            Dat. & puer -\textbf{o}  & puer -\textbf{is} \\
            \hline
            Acc. & puer -\textbf{um} & puer -\textbf{os} \\
            \hline
            Voc. & puer              & puer -\textbf{i} \\
            \hline
            Abl. & puer -\textbf{o}  & puer -\textbf{is} \\
            \hline
        \end{tabular}
        \caption{Conservazione della e (i) nella declinazione}\label{tab:conservazione_ii_decl}
    \end{minipage}
    \hfill
    \begin{minipage}{.4\linewidth}
        \centering
        \begin{tabular}{|c|l|l|}
            \hline
            \textbf{Caso} & \textbf{Singolare} & \textbf{Plurale} \\
            \hline
            Nom. & ager             & agr -\textbf{a} \\
            \hline
            Gen. & agr -\textbf{i}  & agr -\textbf{orum} \\
            \hline
            Dat. & agr -\textbf{o}  & agr -\textbf{is} \\
            \hline
            Acc. & agr -\textbf{um} & agr -\textbf{a} \\
            \hline
            Voc. & ager             & agr -\textbf{a} \\
            \hline
            Abl. & agr -\textbf{o}  & agr -\textbf{is} \\
            \hline
        \end{tabular}
        \caption{Perdita della e nella declinazione}\label{tab:perdita_ii_decl}
    \end{minipage}
\end{table}

\subsection{Eccezioni}
\subsubsection*{Vocativo singolare in -i}
\begin{itemize}
    \item \textit{Filius, i}
    \item \textit{Genius, i}
    \item \textit{Meus, a, um} (aggettivo)
    \item Nomi propri in -\textbf{\u{i}us} (Notare la i breve) - e.g. Tullius
\end{itemize}

\subsubsection*{Genitivo singolare contratto}
\`E possibile trovare una sola i all'uscita del genitivo singolare. E.g. \textit{Vergili} anzich\'e
\textit{Vergilii} (da \textit{Vergilius})

\subsubsection*{Genitivo plurale in -um}
\begin{itemize}
    \item Nomi di monete: \textbf{Sestertium} (\textbf{Sestertiorum})
    \item Composti di \textbf{vir}: \textbf{Triumvirum} (\textbf{Triumvirorum}), \textbf{Duumvirum} (\textbf{Duumvirorum})
    \item Alcuni nomi di popoli: \textbf{Argiri, Argirum}, (\textbf{Argiri, Argirorum})
    \item \textit{Faber}, retto da \textit{praefectus}: \textbf{praefectus fabrum}: Capo dei genieri
    \item \textit{Deus}, in forme speciali: \textbf{ira deum}
    \item \textit{Liberi, Superi, Inferi}
\end{itemize}
\subsubsection*{Declinazione di deus}
\begin{table}[h!]
    \centering
    \begin{tabular}{|c|l|l|l|}
        \hline
        \textbf{Caso} & \textbf{Singolare} & \textbf{Plurale} & \textbf{Pl. Raro}\\
        \hline
        Nom. & deus & di/dii & dei \\
        \hline
        Gen. & dei  & deorum & deum \\
        \hline
        Dat. & deo  & dis/diis & deis \\
        \hline
        Acc. & deum & deos & - \\
        \hline
        Voc. & deus/dive & di/dii & dei \\
        \hline
        Abl. & deo  & dis/diis & deis \\
        \hline
    \end{tabular}
\end{table}

Al vocativo singolare, \textit{deus} viene utilizzato nel latino ecclesiastico, \textit{dive} in quello classico.
\subsubsection*{Neutri in -us}
\textit{Pelagus, virus, vulgus} non hanno il plurale e, nonostante siano neutri hanno nominativo,
accusativo e vocativo in -\textbf{us} anzich\`e in -\textbf{um}.

\begin{table}[h!]
    \centering
    \begin{tabular}{|c|l|l|l|}
        \hline
        \textbf{Caso} & \textbf{Singolare} \\
        \hline
        Nom. & pelag -\textbf{us} \\
        \hline
        Gen. & pelag -\textbf{i} \\
        \hline
        Dat. & pelag -\textbf{o} \\
        \hline
        Acc. & pelag -\textbf{us} \\
        \hline
        Voc. & pelag -\textbf{us} \\
        \hline
        Abl. & pelag -\textbf{o} \\
        \hline
    \end{tabular}
\end{table}

\clearpage
