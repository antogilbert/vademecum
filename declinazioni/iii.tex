\section{III declinazione}
\subsection{Imparisillabi}
Gli imparisillabi sono quei nomi dove il genitivo e il nominativo singolare hanno diverso numero di sillabe.
Caratterizzati dal genitivo plurale in -\textit{um}.
\begin{table}[h!]
    \begin{minipage}{.4\linewidth}
        \centering
        \begin{tabular}{|c|l|l|}
            \hline
            \textbf{Caso} & \textbf{Singolare} & \textbf{Plurale} \\
            \hline
            Nom. & consul              & consul -\textbf{es} \\
            \hline
            Gen. & consul -\textbf{is} & consul -\textbf{um} \\
            \hline
            Dat. & consul -\textbf{i}  & consul -\textbf{ibus} \\
            \hline
            Acc. & consul -\textbf{em} & consul -\textbf{es} \\
            \hline
            Voc. & consul              & consul -\textbf{es} \\
            \hline
            Abl. & consul -\textbf{e}  & consul -\textbf{ibus} \\
            \hline
        \end{tabular}
        \caption{Imparisillabi M \& F}
    \end{minipage}
    \hfill
    \begin{minipage}{.4\linewidth}
        \centering
        \begin{tabular}{|c|l|l|}
            \hline
            \textbf{Caso} & \textbf{Singolare} & \textbf{Plurale} \\
            \hline
            Nom. & nomen              & nomin -\textbf{a} \\
            \hline
            Gen. & nomin -\textbf{is} & nomin -\textbf{um} \\
            \hline
            Dat. & nomin -\textbf{i}  & nomin -\textbf{ibus} \\
            \hline
            Acc. & nomen -\textbf{em} & nomin -\textbf{a} \\
            \hline
            Voc. & nomen              & nomin -\textbf{a} \\
            \hline
            Abl. & nomin -\textbf{i}  & nomin -\textbf{ibus} \\
            \hline
        \end{tabular}
        \caption{Imparisillabi N}
    \end{minipage}
\end{table}

\subsection{Parisillabi}
I parisillabi sono quei nomi dove il genitivo e il nominativo singolare hanno lo stesso numero di sillabe.
Caratterizzati dal genitivo plurale in -\textit{ium} e, per i neutri, dall'ablativo singolare in -\textit{i}.
\begin{table}[h!]
    \begin{minipage}{.4\linewidth}
        \centering
        \begin{tabular}{|c|l|l|}
            \hline
            \textbf{Caso} & \textbf{Singolare} & \textbf{Plurale} \\
            \hline
            Nom. & coll -\textbf{is} & coll -\textbf{es} \\
            \hline
            Gen. & coll -\textbf{is} & coll -\textbf{ium} \\
            \hline
            Dat. & coll -\textbf{i}  & coll -\textbf{ibus} \\
            \hline
            Acc. & coll -\textbf{em} & coll -\textbf{es} \\
            \hline
            Voc. & coll -\textbf{is} & coll -\textbf{es} \\
            \hline
            Abl. & coll -\textbf{e}  & coll -\textbf{ibus} \\
            \hline
        \end{tabular}
        \caption{Parisillabi M \& F}
    \end{minipage}
    \hfill
    \begin{minipage}{.4\linewidth}
        \centering
        \begin{tabular}{|c|l|l|}
            \hline
            \textbf{Caso} & \textbf{Singolare} & \textbf{Plurale} \\
            \hline
            Nom. & mare             & mar -\textbf{ia} \\
            \hline
            Gen. & mar -\textbf{is} & mar -\textbf{ium} \\
            \hline
            Dat. & mar -\textbf{i}  & mar -\textbf{ibus} \\
            \hline
            Acc. & mare             & mar -\textbf{ia} \\
            \hline
            Voc. & mare             & mar -\textbf{ia} \\
            \hline
            Abl. & mar -\textbf{i}  & mar -\textbf{ibus} \\
            \hline
        \end{tabular}
        \caption{Parisillabi N}
    \end{minipage}
\end{table}

\subsection{Eccezioni}
\subsubsection*{Imparisillabi con il genitivo in -ium}
\begin{itemize}
    \item Sostantivi che presentano due consonanti prima dell'uscita -\textit{is} del genitivo singolare
    \item Nomi di popoli in -\textit{as}, -\textit{atis} e -\textit{is}, -\textit{itis}
    \item I seguenti 4 nomi: \textit{Penates, nostrates, vostrates, optimates}
    \item I seguenti monosillabi:
    \begin{itemize}
        \item dos, dotis
        \item faux, faucis
        \item fraus, fraudis
        \item glis, gliris
        \item lis, litis
        \item mas, maris
        \item mus, muris
        \item nix, nivis
        \item strix, strigis
        \item vis, roboris
    \end{itemize}
\end{itemize}

\subsubsection*{Parisillabi con il genitivo in -um}
I seguenti sostantivi:
\begin{itemize}
    \item accipiter, accipitris
    \item pater, patris
    \item mater, matris
    \item frater, fratris
    \item iuvenis, iuvenis
    \item senex, senis
    \item canis, canis
    \item panis, canis
    \item mensis, mensis
    \item sedes, sedis
    \item vates, vatis
    \item volucris, volucris
\end{itemize}

\subsubsection*{Accusativo singolare in -im}
\begin{itemize}
    \item Nomi propri geografici in \textit{is} (e.g. Neapolis)
    \item Sostantivi di origine greca in \textit{is} (e.g. basis)
    \item I seguenti sostantivi:
    \begin{itemize}
        \item amussis, amussis
        \item buris, buris
        \item ravis, ravis
        \item sitis, siti
        \item tussis, tussis
        \item vis, roboris
    \end{itemize}
\end{itemize}

\subsubsection*{Ablativo singolare in -i}
\begin{itemize}
    \item Tutti i nomi della sezione precedente (acc sing in -\textit{im})
    \item I nomi dei mesi
    \item Alcuni nomi che hanno anche l'uscita regolare in -\textit{e}: avis, -is, civis, -is, ignis, -is
\end{itemize}

\subsubsection*{Diversi temi a seconda del caso}
\begin{itemize}
    \item caro, carnis
    \item iter, itineris
    \item Iuppiter, Iovis
    \item senex, senis
    \item suppellex, suppellectilis
\end{itemize}
Nota: Suppellex si usa solo al singolare

\subsubsection*{Nomi con tema doppio}
\begin{itemize}
    \item femur, (femoris, feminis)
    \item iecor, (iecoris, iecinoris)
\end{itemize}

\subsubsection*{Eterocliti}
Gli eterocliti seguono diverse declinazioni a seconda del numero.
\begin{itemize}
    \item vas, vasis $\rightarrow$ Sing. III decl., Pl. II decl.
    \item iugerum, -i $\rightarrow$ Sing. II decl., Pl. III decl.
    \item requies, requietis $\rightarrow$ Acc. Sing. in -\textit{em}, Abl. sing. -\textit{requie}
    \item Neutri Pl. di feste religiose in -\textit{alia} $\rightarrow$ Presentano Gen. Pl. II decl.
    \item Neutri di origine greca in -\textit{ma}, -\textit{matis} $\rightarrow$ Singolare III decl., Plurale II decl.
\end{itemize}

\subsubsection*{Forme irregolari}
\begin{itemize}
    \item bos, bovis $\rightarrow$ Gen. Pl. \textit{boum}, Dat. e Abl. Pl. \textit{bobus}/\textit{bubus}
    \item sus, suis $\rightarrow$ Dat. e Abl. Pl. \textit{subus}
\end{itemize}

\subsubsection*{Declinazione di vis}
\begin{table}[h!]
    \centering
    \begin{tabular}{|c|l|l|}
        \hline
        \textbf{Caso} & \textbf{Singolare} & \textbf{Plurale} \\
        \hline
        Nom. & vis & vires \\
        \hline
        Gen. & roboris  & virium \\
        \hline
        Dat. & robori  & viribus \\
        \hline
        Acc. & vim & vires \\
        \hline
        Voc. & vis & vires \\
        \hline
        Abl. & vi  & viribus \\
        \hline
    \end{tabular}
\end{table}
\clearpage

