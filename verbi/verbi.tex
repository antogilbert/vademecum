\chapter{Verbi}
Il latino ha 4 coniugazioni, distinte dalle uscite dell'infinito. Esse sono (notare la distinzione
negli accenti della 2 e 3 coniugazione):

\begin{enumerate}
    \item -\textbf{are}
    \item -\textbf{\={e}re}
    \item -\textbf{\u{e}re}
    \item -\textbf{ire}
\end{enumerate}

In latino ogni verbo ha il proprio paradigma, consistente delle seguenti 5 voci, in ordine:

\begin{enumerate}
    \item Indicativo presente attivo, 1 pers. singolare
    \item Indicativo presente attivo, 2 pers. singolare
    \item Indicativo perfetto attivo, 1 pers. singolare
    \item Supino attivo
    \item Infinito presente attivo
\end{enumerate}

Esempi di paradigma per le rispettive coniugazioni:
\begin{enumerate}
    \item laudo, laudas, laudavi, laudatum, laudare
    \item moneo, mones, monui, monitum, monere
    \item lego, legis, legi, lectum, legere
    \item audio, audis, audivi, auditum, audire
\end{enumerate}

Il paradigma \`e importante poich\'e i verbi possono essere irregolari, ad esempio:

\begin{itemize}
    \item fero, fers, tuli, latum, ferre
\end{itemize}

\clearpage
\section{Indicativo}

\subsection{Presente}

\begin{table}[h!]
     \centering
     \begin{tabular}{|l|l|l|l|l|l|}
        \hline
        & \textbf{1 Con.} & \textbf{2 Con.} & \textbf{3 Con.} & \textbf{4 Con.} & \textbf{Essere}\\
        \hline
        1 S & laud -o    & mon -eo   & leg -o    & aud -io   & sum \\
        \hline
        2 S & laud -as   & mon -es   & leg -is   & aud -is   & es  \\
        \hline
        3 S & laud -at   & mon -et   & leg -it   & aud -it   & est \\
        \hline
        1 P & laud -amus & mon -emus & leg -imus & aud -imus & sumus\\
        \hline
        2 P & laud -atis & mon -etis & leg -itis & aud -itis & estis\\
        \hline
        3 P & laud -ant  & mon -ent  & leg -unt  & aud -iunt & sunt \\
        \hline
     \end{tabular}
\end{table}

\subsection{Imperfetto}


\begin{table}[h!]
     \centering
     \begin{tabular}{|l|l|l|l|l|l|}
        \hline
         & \textbf{1 Con.} & \textbf{2 Con.} & \textbf{3 Con.} & \textbf{4 Con.} & \textbf{Essere}\\
        \hline
         1 S & laud -abam   & mon -ebam   & leg -ebam   & aud -iebam   & er -am \\
        \hline
         2 S & laud -abas   & mon -ebas   & leg -ebas   & aud -iebas   & er -as \\
        \hline
         3 S & laud -abat   & mon -ebat   & leg -ebat   & aud -iebat   & er -at \\
        \hline
         1 P & laud -abamus & mon -ebamus & leg -ebamus & aud -iebamus & er -amus\\
        \hline
         2 P & laud -abatis & mon -ebatis & leg -ebatis & aud -iebatis & er -atis\\
        \hline
         3 P & laud -abant  & mon -ebant  & leg -ebant  & aud -iebant  & er -ant\\
        \hline
     \end{tabular}
\end{table}

\subsection{Futuro semplice}


\begin{table}[h!]
     \centering
     \begin{tabular}{|l|l|l|l|l|l|}
        \hline
         & \textbf{1 Con.} & \textbf{2 Con.} & \textbf{3 Con.} & \textbf{4 Con.} & \textbf{Essere} \\
        \hline
         1 S & laud -abo    & mon -ebo    & leg -am   & aud -iam   & er -o\\
        \hline
         2 S & laud -abis   & mon -ebis   & leg -es   & aud -ies   & er -is\\
        \hline
         3 S & laud -abit   & mon -ebit   & leg -et   & aud -iet   & er -it\\
        \hline
         1 P & laud -abimus & mon -ebimus & leg -emus & aud -iemus & er -imus\\
        \hline
         2 P & laud -abitis & mon -ebitis & leg -etis & aud -ietis & er -itis\\
        \hline
         3 P & laud -abunt  & mon -ebunt  & leg -ent  & aud -ient  & er -unt\\
        \hline
     \end{tabular}
\end{table}


\subsection{Perfetto}

Per coniugare il perfetto, si prende la radice dal paradigma e si applicano le seguenti desinenze,
indipendentemente dal verbo.

\begin{table}[h!]
     \centering
     \begin{tabular}{|l|l|l|}
        \hline
        & \textbf{Desinenze} & \textbf{Esempio (essere)} \\
        \hline
        1 S & -i     & fu -i     \\
        \hline
        2 S & -isti  & fu -isti  \\
        \hline
        3 S & -it    & fu -it    \\
        \hline
        1 P & -imus  & fu -imus  \\
        \hline
        2 P & -istis & fu -istis \\
        \hline
        3 P & -erunt & fu -erunt \\
        \hline
     \end{tabular}
\end{table}





\clearpage
\section{Imperativo}

\subsection{Presente}

\begin{table}[h!]
     \centering
     \begin{tabular}{|l|l|l|l|l|l|}
        \hline
        & \textbf{1 Con.} & \textbf{2 Con.} & \textbf{3 Con.} & \textbf{4 Con.} & \textbf{Essere}\\
        \hline
        1 S &           &          &          &          & \\
        \hline
        2 S & laud -a   & mon -e   & leg -i   & aud -i   & es \\
        \hline
        3 S &           &          &          &          & \\
        \hline
        1 P &           &          &          &          & \\
        \hline
        2 P & laud -ate & mon -ete & leg -ite & aud -ite & este \\
        \hline
        3 P &           &          &          &          & \\
        \hline
     \end{tabular}
\end{table}




