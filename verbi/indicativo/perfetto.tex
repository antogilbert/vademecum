\subsection{Perfetto}

Per coniugare il perfetto, si prende la radice dal paradigma e si applicano le seguenti desinenze,
indipendentemente dal verbo.

\begin{table}[h!]
     \centering
     \begin{tabular}{|l|l|l|}
        \hline
        & \textbf{Desinenze} & \textbf{Esempio (essere)} \\
        \hline
        1 S & -i     & fu -i     \\
        \hline
        2 S & -isti  & fu -isti  \\
        \hline
        3 S & -it    & fu -it    \\
        \hline
        1 P & -imus  & fu -imus  \\
        \hline
        2 P & -istis & fu -istis \\
        \hline
        3 P & -erunt & fu -erunt \\
        \hline
     \end{tabular}
\end{table}


